\documentclass[a4paper,12pt]{article}

\usepackage[
    left=2.5cm,
    right=2.5cm,
    top=1.5cm,
    bottom=1.5cm,
    headheight=74.2721pt,
    includeheadfoot
]{geometry}

\usepackage{graphicx}
\usepackage{float}
\usepackage{fancyhdr}
\usepackage{multicol}
\usepackage{enumitem}
\usepackage{tabularray}  % already loads xcolor
\UseTblrLibrary{amsmath}
\usepackage{tcolorbox}
\tcbuselibrary{skins}

\usepackage{amsmath}
\usepackage{amsfonts}

\usepackage{pgfplots}
\pgfplotsset{compat=1.18}

\pagestyle{fancy}
\renewcommand{\headrulewidth}{0pt}  % Remove header rule
\fancyhead[L]{
    \includegraphics[width=0.4\linewidth]{IMVheader.png} \\
    \hrulefill \\  % simulate header rule
}
\fancyfoot[C]{}
\fancyfoot[R]{\thepage}
\fancyfoot[L]{\textsuperscript{\textcopyright{}} IMV \the\year{}}  % always up-to-date year

\begin{document}

\centering
{\Huge\bfseries%
    INSTITUTE OF MATHEMATICS \\[0.5\baselineskip]
    VICTORIA \\[0.5\baselineskip]
    \begin{figure}[H]
        \centering
        \includegraphics[width=0.2\linewidth]{IMV yt.png}
    \end{figure}
}

{\LARGE\bfseries
    Mathematical Methods (CAS) U1 \\[0.2cm]
    Written Examination 1
} \\[0.2cm]

{\Large
    Accreditation Period $\sim$ 2023--2027
} \\[1cm]

{\bfseries
    Name: \rule{4cm}{0.4pt}
} \\[1cm]

\begin{table}[H]
    \centering
    \begin{tblr}{width=\linewidth, colspec={X[l, 0.5] *4{X[c]}}, row{1-Z}={m}, hlines, vlines, row{1}={font=\bfseries}}
        Section & {Time \\ Recommended} & {Time \\ Given} & {Marks \\ Allocated} & {Marks \\ Awarded} \\
        A & 35 minutes & 60 minutes & 40 &
    \end{tblr}
\end{table}

\textbf{Instructions}
\begin{itemize}
    \item Detach the formula sheet from the centre of this book during reading time.
    \item Write your student number in the space provided above on this page.
    \item All written responses must be in English.
\end{itemize}

\vspace{0.5cm}
\textbf{Students are NOT permitted access to mobile phones and/or any other unauthorised electronic devices.}

\newpage

% I thought you wanted a double-box, so I added a double box. If not, just remove the borderline= stuff and set the boxrule to 0pt.
\begin{tcolorbox}[enhanced, width=0.95\linewidth, borderline={0.75pt}{-1.5pt}{black}, colframe=black, colback=white, colbacktitle=white, coltitle=black, fonttitle=\bfseries, adjusted title=center, halign title=center, boxrule=0.75pt, titlerule=0mm, toptitle=2.5mm, bottom=4mm, sharp corners=all, title={Instructions}]
    \begin{itemize}[leftmargin=*]  % looks off in tcolorbox setting, so changing left margin
        \item Answer all questions in the spaces provided.
        \item In all questions where a numerical answer is required, an exact value must be given unless otherwise specified.
        \item In questions where more than one mark is available, appropriate working must be shown.
        \item Unless otherwise indicated, the diagrams in this book are not drawn to scale.
    \end{itemize}
\end{tcolorbox}


% Question section
\noindent \textbf{Question 1} \hfill \textbf{(2 marks)} \\[0.3cm]
a) Find the derivative of \( y \) with respect to \( x \) where \( y = x^x \). \hfill \text{(2 marks)} \\[1cm]
\rule{\textwidth}{0.5pt} \\[0.7cm]
\rule{\textwidth}{0.5pt} \\[0.7cm]
\rule{\textwidth}{0.5pt} \\[0.7cm]
\rule{\textwidth}{0.5pt} \\[0.7cm]
\rule{\textwidth}{0.5pt} \\[2cm]
b) Hence, state whether there are any stationary points \hfill \text{(1 marks)} \\[1cm]
\rule{\textwidth}{0.5pt} \\[0.7cm]
\rule{\textwidth}{0.5pt} \\[0.7cm]
\rule{\textwidth}{0.5pt} \\[2cm]


\end{document}
